\documentclass[10pt,conference,compsocconf]{IEEEtran}

\usepackage{hyperref}
\usepackage{graphicx}	% For figure environment
\usepackage{amsmath}

\newcommand{\cN}{\mathcal{N}}
\begin{document}
\title{Writing Scientific Papers and Software}

\author{
  Cheng Soon Ong\\
  \textit{Department of Computer Science, ETH Zurich, Switzerland}
}

\maketitle

\begin{abstract}
  A critical part of scientific discovery is the
  communication of research findings to peers or the general public.
  Mastery of the process of scientific communication improves the
  visibility and impact of research. While this guide is a necessary
  tool for learning how to write in a manner suitable for publication
  at a scientific venue, it is by no means sufficient, on its own, to
  make its reader an accomplished writer. 
  This guide should be a starting point for further development of 
  writing skills.
\end{abstract}

\section{Introduction}

The aim of writing a paper is to infect the mind of your reader with
the brilliance of your idea~\cite{jones08}. 
The hope is that after reading your
paper, the audience will be convinced to try out your idea. In other
words, it is the medium to transport the idea from your head to your
reader's head. 
In the following
section, we show a common structure of scientific papers and briefly
outline some tips for writing good papers in
Section~\ref{sec:tips-writing}.

At that
point, it is important that the reader is able to reproduce your
work~\cite{schwab00,wavelab,gentleman05}. This is why it is also
important that if the work has a computational component, the software
associated with producing the results are also made available in a
useful form. Several guidelines for making your user's experience with
your software as painless as possible is given in
Section~\ref{sec:tips-software}.

This brief guide is by no means sufficient, on its own, to
make its reader an accomplished writer. The reader is urged to use the
references to further improve his or her writing skills.

\section{Models and Methods}
  Describe your idea and how it was implemented to solve
  the problem. Survey the related work, giving credit where credit is
  due.
\section{Results} 
  Show evidence to support your claims made in the
  introduction.
\section{Discussion}
  Discuss the strengths and weaknesses of your
  approach, based on the results. Point out the implications of your
  novel idea on the application concerned.
\section{Summary}
  Summarize your contributions in light of the new
  results.
  
\section{Obtaining Heat-Maps}
In this section we are going to describe the first component of our framework which is a method to transform a given input image into (how we call it) a {\it Road Heat Map}. More precisely suppose an image $I$ is given of dimensions $n\times n$, then the output of our method is an array $x\in [0,1]^{n\times n}$ with the intent that $x(i,j)$ corresponds to the probability that the pixel $(i,j)$ is part of a road. Clearly this is the essential part of the task and the accuracy here crucially affects the final accuracy of our method, no matter how we proceed after that. 

\subsection{Basic CNN}
The provided sample code {\it tf\_aerial\_images.py} of a Convolutional Neural Network for solving this task serves as good starting point for our final approach. Let us describe it briefly here. To compute a prediction for a given input picture of dimension $n\times n$, the picture is first divided into $\frac{n}{16} \times \frac{n}{16}$ chunks of size $16 \times 16$ each. Subsequently, every such chunk $C$ is being input into a neural network $\cN$, which outputs a number $p(C) \in [0,1]$ which is then used as the road heat map value for the corresponding $16\times 16$ chunk. What is left to explain is how is $\cN$ constructed and how is it trained. The network $\cN$ has the following layers: (\textit{Convolutional, RELU, Max Pooling}) repeated twice in this order. To calculate  losses, the cross entropy function is used. The network  is trained using training pairs $(x,y)$ (input and output) of the form: $x$ is a $16\times 16$ chunk, part of a training picture and $y \in \{0,1\}$ is a suitably rounded mean of the corresponding ground truth chunk.  

To deal with the network $\cN$ efficiently, the {\it TensorFlow} library is used. In particular, crucially, this provides us with a ready-to-apply optimization primitive {\it MomentumOptimizer}. We do not go into details on the choice of parameters on which the method is invoked.

\subsection{Big CNN}
Our way of obtaining Heat-Maps builds upon the approach described above, but includes several new insights and modifications which we are going to describe below and briefly discuss how do they affect the results and accuracy.

The first modification we employ concerns the general philosophy how the heat map is obtained. Note that in the basic approach above, the heat map is ``discrete'' in the sense that it is constant on $16 \times 16$ chunks. Anotr disadvantage is that the result for a single chunk depends \emph{only} on the pixels within it, not at all on the neighboring ones, whereas in reality such a prediction cannot be performed locally.

To fix these issues we propose and implement a different method for computing the Heat Map. For every pixel $(i,j)$ in the image we take its square neighborhood of size $48\times 48$ and feed it into a new neural network $\cN'$ to get a single value $p'\in [0,1]$ which determines the value of the corresponding entry $(i,j)$ in the heat map. By this design we get a much ``smoother'' heat map, which also does not suffer so much from the ``locality'' issue, as the value of a pixel is determined by looking at a quite large neighborhood of it. To see a sample on how much improvement does it yield, refer to Figure~\ref{fig:smallbig}.
\begin{figure}
\includegraphics[scale=0.42]{smallbig.png}
\caption{The difference in quality between heat maps computed using the Basic Neural Network A) and the Big Neural Network B).}
\label{fig:smallbig}
\end{figure}
Note that such a change in the model introduces some issues with efficiency. Indeed the network $\cN'$ is significantly larger than $\cN$ and also the number of predictions which needs to be performed increases roughly $16^2=256$ times, when compared to the previous model. Further, also training such a network becomes computationally quite demanding. We obtain reasonable running times by taking advantage of a workstation with 20 cores and using the parallelization power {\it TensorFlow} provides. In particular, for the latter, we coded the predictions to run in batches of $32$, which allows them to be executed in parallel at multiple cores. It turns out that $32$ is essentially the optimal number here, since increasing it to $64$ requires enormous amounts of memory and reducing it to $16$ makes the process a little slower.

\subsection{Diagonal Roads}
One of the issues which appears after training our modelon the training data  only  is that the roads, which are diagonal in the testing data are hardly recognized. This might be before the typical road in the training data is either vertical or horizontal, which is then ``learned'' by the network as correct. For an example see Figure~\ref{fig:diagonal}.
\begin{figure}
\includegraphics[scale=0.3]{diagonal.png}
\caption{The results of predictions when trained on just the training data A) and after adding rotated images B).}
\vspace{2mm}
\label{fig:diagonal}
\end{figure}
To get around this problem we add several random rotations of the training images to the training data. This has the desired effect of making the ``confidence'' of diagonal and non-diagonal predictions equal. Indeed without rotated pictures in the training data, the vertical and horizontal roads were typically detected with a (too) large confidence while the diagonal ones were either very blurry or nonexistent. After this manipulation there is symmetry among different directions of roads in terms of intensity in the heat map.

\subsection{Overfitting}
To control overfitting we were performing validation over $30\%$ of the training data. Indeed we encountered significant overfitting when using different parameters for the model of our Neural Network. When using chunks of size $64$ (instead of $48$ as we use in the final version) the difference between the average prediction scores for instances used for training and for instances outside of the training set reached roughly $0.1$. In the current version it is no more than $0.02$. 

The weak form of validation we use is mainly because of efficiency reasons. While for the Basic Neural Network we were able to run full $K-$fold cross validation for $K=4$, it is no more feasible for the Big Network (because training is very time consuming). Since the results we were obtaining were much better for the Big Network, we decided to stick to it and relax the validation method to a weaker one.



\section{The Post-Processing Phase}

% !TEX root = report.tex

In the previous step, we have obtained, for each pixel $(i,j)$, an estimate $p = p_{ij} \in [0,1]$ of whether this pixel of the (high-resolution) ground truth is white (i.e., a road). We interpret this number as a probability. In this section, we discuss how to convert this to a black or white value for each $16 \times 16$ patch; we call this process \emph{rounding} or \emph{post-processing}. We will treat each patch separately and focus on a single patch.

\subsection{The simple algorithm}

A simple method is just to choose a threshold $t$ (typically $t = 0.25$ since this is how the ground truth is generated)\footnote{In our code, $t$ is called \textit{foreground_threshold}.} and, for each patch, take the mean of the estimated values $m = \frac{\sum p_{ij}}{16 \cdot 16}$. Then, output white iff $m > t$. The parameter $t$ can be optimized to get good validation scores.

% A downside to this approach is that, intuitively, we are discarding the structure given by the patch by just taking its mean. 

\subsection{Integer Programming}

We opt for a more involved approach. First, we estimate the likelihood $\ell \in [0,1]$ of the patch. We would like to take those patches with $\ell > 0.5$. Intuitively, we want it to satisfy that if all pixels have $p \approx 0.25$, then $\ell \approx 0.5$ (since this is the point at which we are not sure whether to take the patch). The same should hold if we have a $0.25$ fraction of pixels with $p = 1$ and the rest have $p = 0$.

One method to get $\ell$ is to compute the mean $m$ and then set $\ell(m)$ to be a piecewise linear function with $\ell(0) = 0, \ell(t) = 0.5, \ell(1) = 1$.

Another one is to assume that each pixel is an \emph{independent} $p$-biased coin flip. Then, we can use dynamic programming to compute, for each $k \in \{0, ... 16 \cdot 16\}$ the exact probability that $k$ pixels are white. We take the probability that the number of white pixels is above the threshold.\footnote{See the code in \texttt{estimate_probability.py} for details, including the recurrence relation for the dynamic programming.}

In practice, we choose the first method because of its simplicity and speed.\footnote{Unfortunately, in Python, this dynamic programming uses about 2-3 seconds per image (even if implemented using NumPy), which makes it difficult to try many hyperparameters.}

Now that we have the likelihood $\ell_v$ for each patch $v$, we develop our Integer Programming approach. Assume that patches are independent. That is, the likelihood of a set of patches is the product of the likelihoods of picking each patch in the set (as well as not picking each patch not in the set). We would like to pick the maximum-likelihood set of patches.

\newcommand{\argmax}{\mathrm{argmax}}
Let $V$ be the set of patches indexed by $v$; thus we want:
\begin{align*}
\argmax_{X \subseteq V} \ell(X) &= \argmax_{X \subseteq V} \prod_{v \in X} \ell_v \prod_{v \in V \setminus X} (1 - \ell_v)
\end{align*}
taking logarithms:
% TODO: please fix this, I dont have time...
\begin{align*}
\argmax_{X \subseteq V} \log \ell(X) &= \argmax_{X \subseteq V} \sum_{v \in X} \log \ell_v + \sum_{v \in V \setminus X} \log (1 - \ell_v) \\
&= \argmax_{X \subseteq V} \sum_{v \in V} \log (1 - \ell_v) + \sum_{v \in X} \log (\ell / (1 - \ell_v)) \\
&= \argmax_{X \subseteq V} \sum_{v \in X} \log (\ell_v / (1 - \ell_v)) \\
&= \argmax_{X \subseteq V} \log (\ell_v / (1 - \ell_v)) x_v
\end{align*}
where $x_v \in \{0,1\}$ denotes whether we choose $v$ or not. Unsurprisingly, this is maximized by taking $x_e = 1$ iff $\log(\ell_v / (1 - \ell_v)) > 0$, which is equivalent to $\ell_v > 0.5$.

However, the fun is only starting: now we can add extra constraints or modify the objective function of this Integer Programming formulation.

For one, we notice that the white region in the ground truth is always connected (at least if we add the border of the image). Indeed, we would like to get rid of disconnected small white segments in our answers, as they are usually wrong.

Second, we want our output to look ``smooth'', without jagged edges (since the ground truth looks like this). We quantify this as follows: we put a penalty on each pair of neighbouring cells which have different colours, thus seeking to minimize the length of the border. Namely, we define a variable $z_{ab} = |x_a - x_b|$ for each two neighbouring cells $a$ and $b$, and add $- \alpha \sum_{ab} z_{ab}$ to the objective function, where $\alpha > 0$ is a parameter to be chosen.\footnote{Called \texttt{border_penalty} in the code.}

For the connectivity, we define a graph: the vertices are the cells, adjacent ones have edges between them, and we add a root vertex which is connected to the border cells. In this graph, we require that there should exist a flow from the root to the set of vertices $v$ with $x_v = 1$, which sinks flow into each such vertex. The flow may only be positive between cells with $x_v = 1$. This can be accomplished by adding linearly many variables and constraints to the program, and enforces that there is no component of white cells disconnected from the root.\footnote{A brief argument for this is as follows: if there was, then the flow must reach these vertices in order to have negative flow conservation, but this is impossible because there is no way to reach this component from the root.}

We can solve this integer program using the Gurobi optimizer.\footnote{It seems to be of just the right size to be feasible, because it is solved in (usually) a few seconds.} This leaves us with optimizing two hyperparameters, which are $t$ and $\alpha$. The best score we are able to get using this method is $0.798$ using the local validation set and $0.917$ on the public Kaggle scoreboard, using a setting of $t = 0.28$ and $\alpha = 0.28$. While this is not a large improvement over the simple method, where we score $0.797$ locally and $0.915$ on Kaggle using $t = 0.28$, we think that this method is interesting and has the potential to yield higher scores if pushed further (perhaps by optimizing hyperparameters when using the dynamic programming method of obtaining the likelihoods $\ell$). A similar method was also used by \cite{??,??} for delineation of curvilinear structures in images (including road networks).
% TODO: please cite:
% https://infoscience.epfl.ch/record/186163
% https://infoscience.epfl.ch/record/201670?ln=en
\footnote{These papers are however somewhat hindered by the size of their Integer Programming formulations, which is quadratic; on the other hand, we managed to obtain a linear-sized formulation, which makes it possible to solve the $38 \times 38$ grid instance in seconds.}






\section*{Acknowledgements}


\bibliographystyle{IEEEtran}
\bibliography{literature}

% TODO: cite this http://www.cs.toronto.edu/~fritz/absps/road_detection.pdf (just that there is a history of doing road semgnetation with NNs)

\end{document}
