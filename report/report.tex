\documentclass[10pt,conference,compsocconf]{IEEEtran}

\usepackage{hyperref}
\usepackage{graphicx}	% For figure environment


\begin{document}
\title{Writing Scientific Papers and Software}

\author{
  Cheng Soon Ong\\
  \textit{Department of Computer Science, ETH Zurich, Switzerland}
}

\maketitle

\begin{abstract}
  A critical part of scientific discovery is the
  communication of research findings to peers or the general public.
  Mastery of the process of scientific communication improves the
  visibility and impact of research. While this guide is a necessary
  tool for learning how to write in a manner suitable for publication
  at a scientific venue, it is by no means sufficient, on its own, to
  make its reader an accomplished writer. 
  This guide should be a starting point for further development of 
  writing skills.
\end{abstract}

\section{Introduction}

The aim of writing a paper is to infect the mind of your reader with
the brilliance of your idea~\cite{jones08}. 
The hope is that after reading your
paper, the audience will be convinced to try out your idea. In other
words, it is the medium to transport the idea from your head to your
reader's head. 
In the following
section, we show a common structure of scientific papers and briefly
outline some tips for writing good papers in
Section~\ref{sec:tips-writing}.

At that
point, it is important that the reader is able to reproduce your
work~\cite{schwab00,wavelab,gentleman05}. This is why it is also
important that if the work has a computational component, the software
associated with producing the results are also made available in a
useful form. Several guidelines for making your user's experience with
your software as painless as possible is given in
Section~\ref{sec:tips-software}.

This brief guide is by no means sufficient, on its own, to
make its reader an accomplished writer. The reader is urged to use the
references to further improve his or her writing skills.

\section{Models and Methods}
  Describe your idea and how it was implemented to solve
  the problem. Survey the related work, giving credit where credit is
  due.
\section{Results} 
  Show evidence to support your claims made in the
  introduction.
\section{Discussion}
  Discuss the strengths and weaknesses of your
  approach, based on the results. Point out the implications of your
  novel idea on the application concerned.
\section{Summary}
  Summarize your contributions in light of the new
  results.


\section*{Acknowledgements}
Agata?

\bibliographystyle{IEEEtran}
\bibliography{literature}

\end{document}
